%%%%%%%%%%%%%%%%%%%%%%%%%%%%%%%%%%%%%%%%%%%%%%%%%%%%%%%%%%%%%%%%%%%%%%%%%%%%%%%%%%%%%%%%%%%
%%
%% LaTeX Template for Faculty of ICT at University of Malta
%%
%% The updated version of this document should be downloaded from
%%      https://github.com/jp-um/university_of_malta_LaTeX_dissertation_template
%%
%% In case of any difficulties please contact Dr Prof. Ebejer on jean.p.ebejer@um.edu.mt
%%
%%%%%%%%%%%%%%%%%%%%%%%%%%%%%%%%%%%%%%%%%%%%%%%%%%%%%%%%%%%%%%%%%%%%%%%%%%%%%%%%%%%%%%%%%%%

%% Before you embark on this quest you should probably read some of:
%% Deadly sins - http://mirrors.ctan.org/info/l2tabu/english/l2tabuen.pdf
%% Writing a thesis in LaTeX - http://tug.org/pracjourn/2008-1/mori/mori.pdf

\RequirePackage[l2tabu, orthodox]{nag} % tells you of any bad LaTeX usage
                                       % must be first thing in class (with the exception of comments)

%% There is one option you should define; oneside or twoside
%% Use twoside for your viva docs (examiners hate long docs they need to carry around)
%% and oneside for the final thing you submit to the library.  Note that margins will
%% change accordingly

\documentclass[oneside]{um-fict}  % custom University of Malta project/dissertation/thesis 


%% **************** (Your) Packages (Start) ******************

% \listfiles % uncomment this to know which packages you are using
              % the list of packages will be in the bottom of the .log file

%% Note that packges may already be loaded from the um (and memoir) classes.
%% Do not add your packages to the template, but rather add them here.

\usepackage{blindtext} %% for some dummy text, remove in your writeup
\usepackage{coffee4}    %% for some fun

%% ***************** (Your) Packages (End) *******************


%% **************** (Your) Data (Start) ******************

\title{Machine Learning Approaches to the Blockchain}  % use \\ here otherwise you get a justified title
                                     % note capitalization of the title (only common 
                                     % words in lower case)
\tagline{some hyped-up tagline}      % tag line
\author{Jean-Paul Ebejer}            % your full name
\authorID{123456M}                   % your University Identifier
\supervisor{Prof.\ Dumbledore}       % your supervisor(s) name
\cosupervisor{Dr Who}                % your cosupervisor(s) name - no . in Dr *OPTIONAL* 
                                     % simply comment out the above line if absent

\degreeName{Some Degree}       		 % the degree you are reading
                                     % note the \ after the dot, so not to consider it a fullstop
\doctype{dissertation}               % the type of document (fyp, dissertation, thesis)
\degreeDate{\monthyeardate\today}    % when did you submit (officially after your corrections)
%%\subjectcode{ICS5200}              % the study unit-code (currently not used)

%% ***************** (Your) Data (End) *******************


%% ******** (Your) Document Settings (Start) *************

% You should have an images directory in every chapX subdir
% NOTE:  Trailing / for subdirs is required.
\graphicspath{{./images/}{./chap1/images/}{./chap2/images/}}   % Paths where to look for images, if defined "images" must always be there as it holds the images in-use by the template.

\makeindex

%% ********* (Your) Document Settings (End) **************

% DOCTOR'S (JP) ORDERS: MAKE SURE TO READ MY TWO BLOG ENTRIES WITH
% CONTENT AND LaTeX TIPS FOR YOUR WRITE-UP.  THESE ARE BASED ON  
% EXAMINER'S FEEDBACK
%
% URLS:
% https://bitsilla.com/blog/2019/03/content-tips-for-your-dissertation-or-project-write-up/
% https://bitsilla.com/blog/2019/01/latex-tips-for-your-dissertation-or-project-write-up/

% end the preamble and start the document

\begin{document}
\frontmatter 
    \maketitle
%%    \input{frontmatter/copyright}       
%%    \input{frontmatter/dedication}        % include a dedication.tex file
    \input{frontmatter/acknowledgements}   % include an acknowledgements.tex file
    %% For tips on how to write a great abstract, have a look at
%%	-	https://www.cdc.gov/stdconference/2018/How-to-Write-an-Abstract_v4.pdf (presentation, start here)
%%	-	https://users.ece.cmu.edu/~koopman/essays/abstract.html
%%	-	https://search.proquest.com/docview/1417403858
%%  - 	https://www.sciencedirect.com/science/article/pii/S037837821830402X

\begin{abstract}
	\blindtext[5]
\end{abstract}

\if@openright\cleardoublepage\else\clearpage\fi
    \tableofcontents*\if@openright\cleardoublepage\else\clearpage\fi
    \listoffigures\if@openright\cleardoublepage\else\clearpage\fi
    \listoftables\if@openright\cleardoublepage\else\clearpage\fi
    \input{frontmatter/abbreviations}\if@openright\cleardoublepage\else\clearpage\fi

%% Note: always use \input as you cannot nest \includes (amongst other things)
%\pagestyle{umpage}
%\floatpagestyle{umpage}
\mainmatter 
    \chapter{Introduction}

Note that you may have multiple \texttt{{\textbackslash}include} statements here, e.g.\ one for each subsection.

General structure of this chapter should read as follows.  This chapter should be used to motivate your study and answer the question ``Why is this important?''. Also, it should define what you set out to achieve (these will be revisited in the conclusions chapter). You should describe your approach to the Aims and Objectives, including an evaluation part. \cofeBm{1}{1}{0}{4cm}{-3cm} %% coffee mark HA HA!

\section{Motivation} % why is this a non trivial problem
\blindtext

\section{Aims and Objectives} 
\blindtext

\section{Our Approach} 
\blindtext

\section{Contributions} 
\blindtext

\section{Document Structure}
\blindtext
 
    \chapter{Background \& Literature Overview}

In this section you need to explain all the theory required to understand your dissertation (i.e.\ the following chapters). But really in this chapter I am going to show you some examples.

\section{Some Technique One}
\index{Some Technique One|(}
\blindtext
\subsection{Some Sub-technique One}
\blindtext
\index{Some Technique One!Some Sub-technique One}
\blindtext
\subsubsection{Some Sub-sub-technique One}
\blindtext
\index{Some Technique One!Some Sub-sub-technique One}
\blindtext
\index{Some Technique One|)}

\section[Some Technique Two]{Some Technique Two with Super Long Title Which Will Overrun In Header}
\index{Some Technique Two|(}
\blindtext[5]

Imagine some colourful description on Some Technique Three\index{Some Technique Three}.

\index{Some Technique Two|)}

\section{Evaluation Criteria}
This section should contain information on the metrics and background used to evaluate your work.

\section{Related Work}
\textbf{In this section you need to explain (and reference) similar work in literature}.  Make sure to:

\begin{itemize}
 \item Give a systematic overview of papers with related/similar work
 \item Highlight similarities/differences to your work (perhaps in the form of a table)
\end{itemize}

Note that this section may be sectioned based on the different aspects of your dissertation.

\section{Summary}
\blindtext[3]

    \chapter{Materials \& Methods}

This section should include a recipe of what you did (explain what you have done so if someone wants to reproduce the experiment, they can).  A flow chart is typically helpful.  Also, make sure to define all software that you used including version numbers and OS.  Should also include a description of statistical methods used (if any).\footnote{For more information see: \url{http://rc.rcjournal.com/content/49/10/1229.short}}

\blindtext

\section{Subsections and Subsubsections}

This is a section.

\subsection{A Subsection}

This is a subsection.

\subsubsection{A Subsubsection}

This is a subsubsection.

\section{Equations}

The following is the most beautiful equation in maths, Euler's Identity (Equation~\ref{eq:euleridentity}).

\begin{equation}\label{eq:euleridentity}
	e^{i\pi}+1=0
\end{equation}
where:
\begin{conditionsenv*}
	e 		& the constant \\
	i 		& of complex fame \\
	\pi		& not of the apple variety \\
\end{conditionsenv*}

\blindtext[2]

\section{Numbered Lists}

This is an example of a numbered list:

\begin{enumerate}
	\item This is my first point
	\item My second
	\item My third!
	\item And my fourth?
\end{enumerate}

\blindtext

\section{Bulleted Lists}

This is an example of a bulleted list:

\begin{itemize}
	\item This is my first point
	\item My second
	\item My third!
	\item And my fourth?
\end{itemize}

\blindtext

\section{Figures}

A test figure is shown in Figure~\ref{fig:test1}.

\begin{figure}[ht!] % supposedly places it here ...
	\centering
	\includegraphics[width=0.6\linewidth]{test_image_goku}
	\caption[This is the short caption for List of Figures]{A test figure.  This caption is huge, but in the list of figures only the smaller version in the square brackets will appear.\index{Goku il-king}}
	\label{fig:test1}
\end{figure}

\blindtext

\section{Two Side-by-Side Figures}

Two figures shown side-by-side are shown in Figure~\ref{fig:test2}.

\begin{figure}[!ht]
	\centering
	\subbottom[Goku]{\includegraphics[width=0.3\textwidth]{test_image_goku}}\qquad
	\subbottom[More Goku]{\includegraphics[width=0.3\textwidth]{test_image_goku}}%
	\caption[Short Caption]{The same super saiyan. Two times.}        
	\label{fig:test2}
\end{figure}

\blindtext

\section{Acronyms}

In the early nineties, \acs{GSM} was deployed in many European countries. \ac{GSM} offered for the first time international roaming for mobile subscribers. The \acs{GSM}’s use of \ac{TDMA} as its communication standard was debated at length. And every now and then there are big discussion whether \ac{CDMA} should have been chosen over \ac{TDMA}.

If you want to know more about \acf{GSM}, \acf{TDMA}, \acf{CDMA} and other acronyms, just read a book about mobile communication. Just to mention it: There is another \ac{UA}, for testing.


\section{Tables}

A beautiful table is shown in Table~\ref{tab:sometable}, data from \textcite{Ebejer2012}.

\begin{table*}\centering
	\ra{1.3}
	\caption{A Beautiful and Complex Table (for tables captions above)}\label{tab:sometable}	
	\begin{tabular}{@{}rrrrcrrr@{}}\toprule
		& \multicolumn{3}{c}{$w = 8$} & \phantom{abc}& \multicolumn{3}{c}{$w = 16$} \\
		\cmidrule{2-4} \cmidrule{6-8} 
		& $t=0$ & $t=1$ & $t=2$ && $t=0$ & $t=1$ & $t=2$\\ \midrule
		$dir=1$\\
		$c$ & 0.0790 & 0.1692 & 0.2945 && 0.3670 & 0.7187 & 3.1815\\
		$c$ & -0.8651& 50.0476& 5.9384&& -9.0714& 297.0923& 46.2143\\
		$c$ & 124.2756& -50.9612& -14.2721&& 128.2265& -630.5455& -381.0930\\
		$dir=0$\\
		$c$ & 0.0357& 1.2473& 0.2119&& 0.3593& -0.2755& 2.1764\\
		$c$ & -17.9048& -37.1111& 8.8591&& -30.7381& -9.5952& -3.0000\\
		$c$ & 105.5518& 232.1160& -94.7351&& 100.2497& 141.2778& -259.7326\\
		\bottomrule
	\end{tabular}
\end{table*}

\blindtext

\section{Long Tables}

The following is an example of a table (Table~\ref{tab:full_dude_results}) spanning multiple pages.

\newcolumntype{P}[1]{>{\centering\arraybackslash}p{#1}}
\begin{center}
	\begingroup
	\renewcommand\arraystretch{0.66} % only applicable to this table because of group
	
	\begin{longtable}{lrrrrrrrr}
		\caption[Performance of Ligity in HTS mode against the Ligity-compatible DUD-E targets]{Performance of Ligity in HTS mode against the Ligity-compatible DUD-E targets. The mean (and standard deviation in parentheses) values of ROC AUC using Tanimoto is 0.622 ($\pm 0.132$), while for Tversky it is 0.671 ($\pm 0.142$); the mean EF\textsubscript{1\%} using Tanimoto is 5.648 ($\pm 8.668$), while for EF\textsubscript{1\%} using Tversky it is 9.047 ($\pm 12.713$).} 
		\label{tab:full_dude_results} 
		\\ 
		\toprule 
		\multicolumn{1}{l}{\textbf{Target}}
		& \multicolumn{1}{P{1cm}}{\textbf{No.\ of Actives}}
		& \multicolumn{1}{P{1cm}}{\textbf{No.\ of Decoys}} 
		& \multicolumn{1}{P{1.25cm}}{\textbf{ROC AUC Tanimoto}} 
		& \multicolumn{1}{P{1.25cm}}{\textbf{ROC AUC Tversky}} 
		& \multicolumn{1}{P{1.25cm}}{\textbf{BEDROC Tanimoto}} 
		& \multicolumn{1}{P{1.25cm}}{\textbf{BEDROC Tversky}} 
		& \multicolumn{1}{P{1.25cm}}{\textbf{EF\textsubscript{1\%} Tanimoto}} 
		& \multicolumn{1}{P{1.5cm}}{\textbf{EF\textsubscript{1\%} Tversky}}\\	
		\midrule
		\endfirsthead
		\midrule
		\multicolumn{1}{l}{\textbf{Target}}
		& \multicolumn{1}{P{1cm}}{\textbf{No.\ of Actives}}
		& \multicolumn{1}{P{1cm}}{\textbf{No.\ of Decoys}} 
		& \multicolumn{1}{P{1.25cm}}{\textbf{ROC AUC Tanimoto}} 
		& \multicolumn{1}{P{1.25cm}}{\textbf{ROC AUC Tversky}}
		& \multicolumn{1}{P{1.25cm}}{\textbf{BEDROC Tanimoto}} 
		& \multicolumn{1}{P{1.25cm}}{\textbf{BEDROC Tversky}} 
		& \multicolumn{1}{P{1.25cm}}{\textbf{EF\textsubscript{1\%} Tanimoto}} 
		& \multicolumn{1}{P{1.25cm}}{\textbf{EF\textsubscript{1\%} Tversky}}\\	
		\midrule	
		\endhead
		\midrule	
		\multicolumn{7}{r@{}}{(continued\ldots)}\\
		\endfoot
		\endlastfoot
		ABL1   & 182   & 10,750   & 0.563   & 0.473   & 0.077   & 0.077   & 1.653   & 2.204  \\
		ACE    & 281   & 16,877   & 0.787   & 0.787   & 0.336   & 0.401   & 12.425  & 19.525 \\
		ACES   & 453   & 26,242   & 0.634   & 0.645   & 0.077   & 0.155   & 1.766   & 5.518  \\
		ADA    & 93    & 5,450    & 0.724   & 0.660   & 0.149   & 0.147   & 3.251   & 3.251  \\
		ADA17  & 532   & 35,898   & 0.638   & 0.728   & 0.103   & 0.283   & 1.317   & 9.030  \\
		ADRB1  & 247   & 15,850   & 0.523   & 0.647   & 0.065   & 0.129   & 1.619   & 5.262  \\
		ADRB2  & 231   & 14,999   & 0.523   & 0.589   & 0.052   & 0.040   & 1.735   & 0.000  \\
		AKT1   & 293   & 16,450   & 0.386   & 0.548   & 0.039   & 0.107   & 2.737   & 3.080  \\
		AKT2   & 117   & 6,900    & 0.511   & 0.685   & 0.140   & 0.194   & 8.568   & 8.568  \\
		ALDR   & 159   & 8,988    & 0.574   & 0.610   & 0.202   & 0.172   & 10.747  & 6.322  \\
		AMPC   & 48    & 2,845    & 0.521   & 0.541   & 0.049   & 0.023   & 0.000   & 0.000  \\
		ANDR   & 269   & 14,349   & 0.722   & 0.742   & 0.194   & 0.354   & 4.839   & 24.938 \\
		AOFB   & 121   & 6,875    & 0.422   & 0.464   & 0.045   & 0.027   & 1.652   & 0.000  \\
		BACE1  & 283   & 18,100   & 0.441   & 0.775   & 0.017   & 0.310   & 0.000   & 13.062 \\
		BRAF   & 152   & 9,950    & 0.612   & 0.639   & 0.208   & 0.165   & 12.502  & 5.264  \\
		CASP3  & 199   & 10,694   & 0.600   & 0.734   & 0.068   & 0.258   & 0.502   & 7.031  \\
		CDK2   & 474   & 27,838   & 0.467   & 0.507   & 0.021   & 0.048   & 0.000   & 1.055  \\
		COMT   & 41    & 3,846    & 0.789   & 0.889   & 0.338   & 0.665   & 19.447  & 58.341 \\
		CP2C9  & 120   & 7,449    & 0.518   & 0.634   & 0.058   & 0.186   & 1.660   & 8.299  \\
		CP3A4  & 170   & 11,787   & 0.450   & 0.493   & 0.022   & 0.057   & 0.000   & 2.345  \\
		CSF1R  & 166   & 12,149   & 0.526   & 0.542   & 0.136   & 0.152   & 6.031   & 7.238  \\
		CXCR4  & 40    & 3,405    & 0.575   & 0.722   & 0.217   & 0.134   & 12.665  & 0.000  \\
		DEF    & 102   & 5,699    & 0.732   & 0.833   & 0.212   & 0.379   & 10.786  & 15.689 \\
		DHI1   & 330   & 19,348   & 0.481   & 0.595   & 0.089   & 0.062   & 2.422   & 1.211  \\
		DPP4   & 533   & 40,941   & 0.586   & 0.591   & 0.154   & 0.157   & 4.312   & 3.937  \\
		DRD3   & 480   & 34,048   & 0.484   & 0.441   & 0.043   & 0.046   & 1.251   & 0.626  \\
		DYR    & 231   & 17,196   & 0.694   & 0.758   & 0.210   & 0.230   & 6.504   & 7.371  \\
		EGFR   & 542   & 35,047   & 0.593   & 0.491   & 0.054   & 0.037   & 0.922   & 0.000  \\
		ESR1   & 383   & 20,683   & 0.838   & 0.861   & 0.527   & 0.594   & 31.281  & 39.101 \\
		ESR2   & 367   & 20,199   & 0.844   & 0.870   & 0.563   & 0.644   & 20.130  & 32.644 \\
		FA10   & 537   & 28,324   & 0.564   & 0.674   & 0.058   & 0.118   & 0.930   & 2.232  \\
		FA7    & 114   & 6,249    & 0.762   & 0.859   & 0.210   & 0.332   & 6.105   & 8.721  \\
		FABP4  & 47    & 2,749    & 0.786   & 0.744   & 0.191   & 0.276   & 0.000   & 10.623 \\
		FAK1   & 100   & 5,350    & 0.642   & 0.531   & 0.111   & 0.065   & 2.019   & 0.000  \\
		FGFR1  & 139   & 8,698    & 0.511   & 0.522   & 0.036   & 0.088   & 0.722   & 1.445  \\
		FKB1A  & 111   & 5,799    & 0.605   & 0.751   & 0.162   & 0.164   & 8.122   & 3.610  \\
		FNTA   & 592   & 51,493   & 0.411   & 0.625   & 0.012   & 0.132   & 0.000   & 4.053  \\
		FPPS   & 85    & 8,842    & 0.917   & 0.985   & 0.323   & 0.776   & 2.360   & 36.581 \\
		GCR    & 258   & 14,998   & 0.805   & 0.834   & 0.244   & 0.324   & 3.092   & 8.116  \\
		GLCM   & 54    & 3,790    & 0.667   & 0.685   & 0.182   & 0.279   & 1.873   & 11.240 \\
		GRIA2  & 158   & 11,842   & 0.662   & 0.684   & 0.248   & 0.154   & 11.392  & 5.696  \\
		GRIK1  & 101   & 6,547    & 0.656   & 0.668   & 0.203   & 0.102   & 7.978   & 1.995  \\
		HDAC2  & 185   & 10,300   & 0.676   & 0.734   & 0.187   & 0.201   & 4.318   & 4.318  \\
		HDAC8  & 170   & 10,449   & 0.640   & 0.819   & 0.120   & 0.377   & 2.946   & 8.250  \\
		HIVINT & 100   & 6,640    & 0.390   & 0.554   & 0.030   & 0.116   & 0.000   & 3.018  \\
		HIVPR  & 535   & 35,724   & 0.663   & 0.872   & 0.072   & 0.490   & 0.187   & 23.898 \\
		HIVRT  & 338   & 18,884   & 0.495   & 0.475   & 0.124   & 0.085   & 4.443   & 1.777  \\
		HMDH   & 170   & 8,750    & 0.480   & 0.906   & 0.068   & 0.652   & 2.358   & 35.963 \\
		HS90A  & 88    & 4,850    & 0.635   & 0.506   & 0.096   & 0.083   & 0.000   & 3.436  \\
		HXK4   & 92    & 4,700    & 0.662   & 0.803   & 0.206   & 0.307   & 15.192  & 9.766  \\
		IGF1R  & 148   & 9,300    & 0.502   & 0.575   & 0.057   & 0.189   & 2.037   & 14.941 \\
		INHA   & 43    & 2,300    & 0.493   & 0.575   & 0.031   & 0.045   & 0.000   & 0.000  \\
		ITAL   & 138   & 8,500    & 0.619   & 0.465   & 0.037   & 0.065   & 0.000   & 0.728  \\
		JAK2   & 107   & 6,500    & 0.472   & 0.475   & 0.073   & 0.118   & 2.807   & 6.549  \\
		KIF11  & 116   & 6,850    & 0.755   & 0.781   & 0.149   & 0.219   & 4.289   & 2.574  \\
		KIT    & 166   & 10,449   & 0.463   & 0.437   & 0.045   & 0.030   & 0.000   & 0.000  \\
		KITH   & 57    & 2,850    & 0.649   & 0.838   & 0.228   & 0.709   & 14.069  & 47.483 \\
		KPCB   & 135   & 8,699    & 0.753   & 0.813   & 0.220   & 0.338   & 8.923   & 12.641 \\
		LCK    & 419   & 27,391   & 0.471   & 0.437   & 0.031   & 0.043   & 0.000   & 1.910  \\
		LKHA4  & 171   & 9,448    & 0.718   & 0.694   & 0.238   & 0.150   & 8.203   & 1.758  \\
		MAPK2  & 101   & 6,148    & 0.660   & 0.670   & 0.174   & 0.199   & 5.988   & 3.992  \\
		MCR    & 94    & 5,149    & 0.816   & 0.888   & 0.215   & 0.454   & 6.436   & 19.307 \\
		MET    & 166   & 11,249   & 0.566   & 0.531   & 0.130   & 0.065   & 6.032   & 0.603  \\
		MK01   & 79    & 4,550    & 0.518   & 0.602   & 0.121   & 0.206   & 5.095   & 3.821  \\
		MK10   & 104   & 6,600    & 0.488   & 0.489   & 0.020   & 0.031   & 0.962   & 0.962  \\
		MK14   & 578   & 35,847   & 0.511   & 0.589   & 0.040   & 0.064   & 0.173   & 0.519  \\
		MMP13  & 572   & 37,199   & 0.648   & 0.753   & 0.134   & 0.268   & 2.446   & 9.957  \\
		MP2K1  & 121   & 8,146    & 0.669   & 0.569   & 0.187   & 0.058   & 3.293   & 0.823  \\
		NOS1   & 98    & 8,028    & 0.483   & 0.451   & 0.109   & 0.041   & 3.071   & 0.000  \\
		NRAM   & 98    & 6,200    & 0.853   & 0.859   & 0.342   & 0.290   & 11.221  & 3.060  \\
		PA2GA  & 99    & 5,150    & 0.793   & 0.756   & 0.225   & 0.153   & 1.020   & 3.059  \\
		PARP1  & 508   & 30,029   & 0.635   & 0.692   & 0.215   & 0.231   & 11.234  & 7.884  \\
		PGH1   & 195   & 10,798   & 0.645   & 0.637   & 0.077   & 0.100   & 0.000   & 2.050  \\
		PGH2   & 435   & 23,139   & 0.716   & 0.780   & 0.166   & 0.291   & 3.444   & 9.874  \\
		PLK1   & 107   & 6,800    & 0.658   & 0.531   & 0.123   & 0.048   & 1.871   & 0.000  \\
		PNPH   & 103   & 6,946    & 0.575   & 0.578   & 0.161   & 0.181   & 4.888   & 8.799  \\
		PPARA  & 373   & 19,399   & 0.783   & 0.778   & 0.262   & 0.280   & 6.693   & 7.764  \\
		PPARD  & 240   & 12,250   & 0.547   & 0.544   & 0.078   & 0.098   & 1.665   & 2.498  \\
		PPARG  & 484   & 25,299   & 0.515   & 0.605   & 0.055   & 0.118   & 0.619   & 4.955  \\
		PRGR   & 293   & 15,648   & 0.740   & 0.793   & 0.142   & 0.318   & 2.053   & 14.714 \\
		PTN1   & 130   & 7,249    & 0.398   & 0.538   & 0.055   & 0.090   & 0.000   & 3.068  \\
		PUR2   & 50    & 2,700    & 0.851   & 0.837   & 0.281   & 0.255   & 7.857   & 1.964  \\
		PYGM   & 77    & 3,944    & 0.403   & 0.492   & 0.016   & 0.137   & 0.000   & 3.917  \\
		PYRD   & 111   & 6,449    & 0.682   & 0.710   & 0.462   & 0.413   & 34.027  & 16.118 \\
		RENI   & 104   & 6,956    & 0.720   & 0.789   & 0.043   & 0.138   & 0.000   & 0.000  \\
		ROCK1  & 100   & 6,300    & 0.347   & 0.449   & 0.020   & 0.084   & 1.000   & 4.000  \\
		RXRA   & 131   & 6,950    & 0.788   & 0.900   & 0.219   & 0.596   & 6.091   & 27.407 \\
		SAHH   & 63    & 3,450    & 0.874   & 0.852   & 0.598   & 0.542   & 35.050  & 27.084 \\
		SRC    & 524   & 34,500   & 0.565   & 0.477   & 0.065   & 0.050   & 0.382   & 0.573  \\
		TGFR1  & 133   & 8,499    & 0.609   & 0.639   & 0.147   & 0.154   & 10.565  & 4.528  \\
		THB    & 103   & 7,450    & 0.794   & 0.762   & 0.238   & 0.150   & 10.614  & 0.965  \\
		THRB   & 461   & 27,000   & 0.605   & 0.706   & 0.063   & 0.166   & 2.166   & 5.632  \\
		TRY1   & 449   & 25,975   & 0.711   & 0.815   & 0.147   & 0.280   & 2.898   & 6.688  \\
		TRYB1  & 148   & 7,650    & 0.670   & 0.670   & 0.153   & 0.132   & 3.378   & 3.378  \\
		TYSY   & 109   & 6,745    & 0.594   & 0.725   & 0.071   & 0.226   & 0.911   & 5.468  \\
		UROK   & 162   & 9,850    & 0.525   & 0.650   & 0.036   & 0.120   & 0.000   & 1.854  \\
		VGFR2  & 409   & 24,948   & 0.632   & 0.578   & 0.083   & 0.093   & 1.465   & 1.465  \\
		WEE1   & 102   & 6,150    & 0.934   & 0.929   & 0.789   & 0.797   & 59.348  & 61.294 \\
		XIAP   & 100   & 5,150    & 0.752   & 0.974   & 0.190   & 0.897   & 8.077   & 51.490 \\
		\bottomrule
	\end{longtable}
	
	\endgroup
\end{center}

\blindtext

\section{Landscape Tables}
Next is an example of a wide table on a landscape oriented paper (Table~\ref{tab:land}).

\begin{landscape}
	\pagestyle{empty} %% only if you want to remove silly headers on the side	
	\begin{table}[h]
		\caption[A landscape table]{A table in landscape orientation.} 
		\begin{tabular}{rrrrrrrrrrrrrrr} \toprule
			\label{tab:land} 		
			{$m$} & {$x$} & {$y$} & {$z$} & {$a$} & {$A_m$} & {$B$} & {$C$} & {$x$} & {$y$} & {$z$} & {$a$} & {$A_m$} & {$B$} & {$C$} \\ \midrule
			1  & 16.128 & +8.872 & 16.128 & 1.402 & 1.373 & -146.6 & -137.6 & 16.128 & +8.872 & 16.128 & 1.402 & 1.373 & -146.6 & -137.6\\
			2  & 3.442  & -2.509 & 3.442  & 0.299 & 0.343 & 133.2  & 152.4 & 3.442  & -2.509 & 3.442  & 0.299 & 0.343 & 133.2  & 152.4 \\
			3  & 1.826  & -0.363 & 1.826  & 0.159 & 0.119 & 168.5  & -161.1 & 1.826  & -0.363 & 1.826  & 0.159 & 0.119 & 168.5  & -161.1 \\
			4  & 0.993  & -0.429 & 0.993  & 0.086 & 0.08  & 25.6   & 90 & 1.826  & -0.363 & 1.826  & 0.159 & 0.119 & 168.5  & -161.1    \\ \midrule
			5  & 1.29   & +0.099 & 1.29   & 0.112 & 0.097 & -175.6 & -114.7 & 1.826  & -0.363 & 1.826  & 0.159 & 0.119 & 168.5  & -161.1\\
			6  & 0.483  & -0.183 & 0.483  & 0.042 & 0.063 & 22.3   & 122.5 & 1.826  & -0.363 & 1.826  & 0.159 & 0.119 & 168.5  & -161.1 \\
			7  & 0.766  & -0.475 & 0.766  & 0.067 & 0.039 & 141.6  & -122 & 1.826  & -0.363 & 1.826  & 0.159 & 0.119 & 168.5  & -161.1  \\
			8  & 0.624  & +0.365 & 0.624  & 0.054 & 0.04  & -35.7  & 90  & 1.826  & -0.363 & 1.826  & 0.159 & 0.119 & 168.5  & -161.1   \\ \midrule
			9  & 0.641  & -0.466 & 0.641  & 0.056 & 0.045 & 133.3  & -106.3 & 1.826  & -0.363 & 1.826  & 0.159 & 0.119 & 168.5  & -161.1\\
			10 & 0.45   & +0.421 & 0.45   & 0.039 & 0.034 & -69.4  & 110.9  & 1.826  & -0.363 & 1.826  & 0.159 & 0.119 & 168.5  & -161.1\\
			11 & 0.598  & -0.597 & 0.598  & 0.052 & 0.025 & 92.3   & -109.3 & 1.826  & -0.363 & 1.826  & 0.159 & 0.119 & 168.5  & -161.1\\ \bottomrule
		\end{tabular}
	\end{table}
\end{landscape}

\blindtext

\section{Theorems}

\begin{theorem}
	Let \(f\) be a function whose derivative exists in every point, then \(f\) is 
	a continuous function.
\end{theorem}

\begin{theorem}[Pythagorean theorem]
	\label{pythagorean}
	This is a theorem about right triangles and can be summarised in the next 
	equation 
	\[ x^2 + y^2 = z^2 \]
\end{theorem}

And a consequence of Theorem \ref{pythagorean} is the statement in the next corollary.

\begin{corollary}
	There's no right rectangle whose sides measure 3~cm, 4~cm, and 6~cm.
\end{corollary}

You can reference theorems such as \ref{pythagorean} when a label is assigned.

\blindtext

\section{Lemmas}

\begin{lemma}
	Given two line segments whose lengths are \(a\) and \(b\) respectively there is a 
	real number \(r\) such that \(b=ra\).
\end{lemma}

\blindtext

\section{Proofs}

\begin{lemma}
	Given two line segments whose lengths are \(a\) and \(b\) respectively there 
	is a real number \(r\) such that \(b=ra\).
\end{lemma}

\begin{proof}
	To prove it by contradiction try and assume that the statement is false,
	proceed from there and at some point you will arrive to a contradiction.
\end{proof}

\blindtext

\section{Code Listings}

Here you go.

\begin{lstlisting}[language=Python,caption={My Listing Caption},captionpos=b]
	import numpy as np
	
	def incmatrix(genl1,genl2):
	m = len(genl1)
	n = len(genl2)
	M = None #to become the incidence matrix
	VT = np.zeros((n*m,1), int)  #dummy variable
	
	#compute the bitwise xor matrix
	M1 = bitxormatrix(genl1)
	M2 = np.triu(bitxormatrix(genl2),1) 
	
	for i in range(m-1):
	for j in range(i+1, m):
	[r,c] = np.where(M2 == M1[i,j])
	for k in range(len(r)):
	VT[(i)*n + r[k]] = 1;
	VT[(i)*n + c[k]] = 1;
	VT[(j)*n + r[k]] = 1;
	VT[(j)*n + c[k]] = 1;
	
	if M is None:
	M = np.copy(VT)
	else:
	M = np.concatenate((M, VT), 1)
	
	VT = np.zeros((n*m,1), int)
	
	return M
\end{lstlisting}

\blindtext

\section{Algorithms}


%% This is needed if you want to add comments in
%% your algorithm with \Comment
\SetKwComment{Comment}{/* }{ */}

\begin{algorithm}[hbt!]
	\caption{An algorithm with caption}\label{alg:example}
	\KwData{$n \geq 0$}
	\KwResult{$y = x^n$}
	$y \gets 1$\;
	$X \gets x$\;
	$N \gets n$\;
	\While{$N \neq 0$}{
		\eIf{$N$ is even}{
			$X \gets X \times X$\;
			$N \gets \frac{N}{2} $ \Comment*[r]{This is a comment}
		}{\If{$N$ is odd}{
				$y \gets y \times X$\;
				$N \gets N - 1$\;
			}
		}
	}
\end{algorithm}

An algorithm example is shown in Algorithm~\ref{alg:example}. \blindtext

\section{Suppressing Page Numbers on a Float Page}

Kindly refer to Figure~\ref{fig:largegoku}. 

\blindtext

\begin{figure}[!ht]
	\thisfloatpagestyle{empty} %% This is the key line to suppress page decorations (including nos.) on float pages.
	\centering
	\includegraphics[width=0.9\textwidth]{goku-large}
	\caption[Short Random Caption]{Page numbers are suppressed on this page.}
	\label{fig:largegoku}
\end{figure}

\blindtext
\FloatBarrier % from placeins

\section{Referencing}

Use \texttt{\textbackslash textcite} for in-text citations, e.g.\ \textcite{Einstein1905}, and \texttt{\textbackslash parencite} for citations in parenthesis. 

In their study, \textcite{Einstein1905} show the world is round. Others have shown this to be the case \parencite{Arrighi2003, Ebejer2016}.

\section{Summary}
\blindtext[5]
    \chapter{Results \& Discussion}

Should include a reiteration of the experiments, and their outcome.  Together with a description (discussion).  Preamble should include a reminder of the aims and objectives together with a list of experiments to achieve these.  Should include many charts and other visualization with appropriate descriptions.\footnote{Another footnote example.}  

\blindtext

\section{Summary}

\blindtext
    \chapter{Evaluation}

In an ideal world, you should have two kind of evaluations. The first is against some ground truth (perhaps a random model?). The second kind of evaluation is against other people's work (accuracy, speed, etc.). Any dimension which is of interest, should be evaluated.  Evaluation should be statistically sound.

\blindtext

\section{Summary}
\blindtext
    \input{chap6/conclusions_main}

%%\pagestyle{umpageback}
{%backmatter % comment this out otherwise are not numbered
    % Bibliography
    \if@openright\cleardoublepage\else\clearpage\fi
	%% For references use IEEE style [5] or Harvard style [6]
    \bibliographystyle{um-plainnat} %% specific plainnat does not show url for articles
    % Use something like https://flamingtempura.github.io/bibtex-tidy/ to clean all your bibtex entries
    {\scriptsize\bibliography{chap1/introduction_biblio,chap2/background_and_lit_overview_biblio}}
	\printindex
}

\appendix
	\chapter{Media Content}

If the dissertation has a DVD or pendrive attached to it, you will need a section which explains what is on the media (structure, files, data, etc.).  This could be a table with filename and description.

\blindtext
     % these are just test names as I didn't know what you'd want
	\chapter{Installation Instructions}
\blindtext
    
	\chapter{User Manual}
\blindtext
 




\end{document}

%%% The End %%%
